%****************************************************************************/
%
% File:		modules_des.tex
%
% Project:	xdevmenu application documentation
%
% Description:	"Modules description" section
%
% Author(s):	Faranguiss Poncet
%
% Original:	March 1991
%
%
% $Header: /home/jkrueger1/sources/taco/backup/taco/system/xdevmenu/doc/modules_des.tex,v 1.2 2003-12-17 14:24:25 jkrueger1 Exp $
%
% Copyright (c) 1990-2014 by European Synchrotron Radiation Facility,
%                       Grenoble, France
%
%		       All Rights Reserved
%
%****************************************************************************/
As described previously (section 2.1.2 Internal architecture), Xdevmenu is 
composed of three C code modules and one uil file.

In this section we are going to describe the role of each of these modules,
present the data structures defined within the module, and give a short
description of some of the most important functions defined within the module.

\subsubsection{xdev\_global.h}
This module contains one global constant and four global variables for all the 
application~:
\begin {itemize}
\item {\bf UIDFILENAME : } 
the location concatenated with the name of the created uid 
file for the access to that file at run time.

\item {\bf toplevel : } 
the top most widget created by the "main" function using Xtoolkit.

\item {\bf dev\_selection\_dial : } 
this widget allows the selection of the device to be opened. It has to be known in 
different modules of the application since the user can open ( and therefore select ) 
dynamically another device using the option "open device". 

\item {\bf error\_dial\_widget : } 
this widget allows an error dialogue window\footnote[1]{ OSF/MOTIF concept : see OSF/MOTIF programmer's guide } 
to be displayed. It has to be known in different modules of the application because
an error dialogue window is asked to be displayed in any of the modules.

\item {\bf currentDevMask : } this string contains the current pathname of the devices' location.

\item {\bf currentDevFilMask : } 
this string contains the subdirectory filter which can be associated to the current pathname of the devices' location.
\end {itemize}

\newpage
\subsubsection{The uil file (xdevmenu.uil)}
\input uil_file

\newpage
\subsubsection{xdevmenu.c}
\input xdev_c

\newpage
\subsubsection{dev\_gui.c}
\input dev_gui

\newpage
\subsubsection{dev\_interface.c}
\input dev_interface
