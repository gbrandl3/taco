%****************************************************************************/
%
% File:		dev_interface.tex
%
% Project:	xdevmenu application documentation
%
% Description:	"dev_interface.c" section
%
% Author(s):	Faranguiss Poncet
%
% Original:	March 1991
%
%
% $Header: /home/jkrueger1/sources/taco/backup/taco/system/xdevmenu/doc/dev_interface.tex,v 1.1 2003-04-25 12:54:29 jkrueger1 Exp $
%
% Copyright (c) 1990-2013 by European Synchrotron Radiation Facility, 
%                       Grenoble, France
%
%		       All Rights Reserved
%
%****************************************************************************/
This module makes the interface between the rest of the application code
and the device server library calls.

Another module which needs to call a device server function calls a function
in this module which in turn will call device server routines if necessary after
some transformation of data structure.

This module contains a data structure definition.

%\sy{1}	{Internal data structures and variables}
\paragraph{Internal data structures and variables : }

\begin{itemize}
\item
{\bf device\_entry\_structure : } the most important data structure defined in
this module is "device\_entry\_structure". This data structure is used to store
all the information about the device which could be used later on by the
application. This information includes the device reference (returned by
the device server when opening the connection to a device) and the command
data structure. The command data structure contains all the command names
available for a device and for each of them the information about input and
output arguments.
This data structure is useful because instead of asking for the list of
available commands each time the user wants to execute a command on a device,
this information is returned by the device server only once by the query command
called when opening the connection to the device. This information is then
stored in the data structure for the further use.
\item
{\bf device\_table : } it is the table in which each entry is a
device\_entry\_structure, storing the above information for one device.
\end{itemize}

%\xs{1}

%\sy{1}	{get\_device\_type}
\paragraph{get\_device\_type : }

This function is only a simulation function. Its purpose is to return the
type of the device given its name. Since, there is no routine in the device
server library providing this function, we had to code this function.

Therefore this function will disappear, as soon as the device server library
provides the equivalent function. 

%\xs{1}

%\sy{1}	{open\_selected\_device}
\paragraph{open\_selected\_device : }

This function is in charge of opening the selected device and also
initializing all the internal data structures associated with that device.

To perform its task, this function first calls the "dev\_import" routine to
open the connection to the device server for the selected device. Once this
routine has been performed successfully, this function calls the
"dev\_cmd\_query" routine to get the list of available commands (and their
arguments) for that device. Finally it stores all this information in one
entry of the "device\_table" for the further use.

%\xs{1}

%\sy{1}	{get\_list\_cmd\_names}
\paragraph{get\_list\_cmd\_names : }

This function returns the number of commands and also the list of available
commands for a device.

It searches through the "device\_table" and finds the entry corresponding to the
device whose name is passed as the input parameter. It then retrieves the
information concerning the command names and the number of commands.

%\xs{1}

%\sy{1}	{get\_cmd\_in\_parameter}
\paragraph{get\_cmd\_in\_parameter : }

This function returns the name of a command's input argument and its data type.
The input parameters for this function are the device name and the command
name.

This function searches through the "device\_table" to get the information. 
Hence, the device server is not called.

%\xs{1}

%\sy{1}	{dev\_exec\_cmd}
\paragraph{dev\_exec\_cmd : }

This function is called to execute a command on a device. The input argument
of the command is provided if it exists. The function calls the
"dev\_put\_get" routine in the device server library, to execute the command
and to get (if it exists) the output argument. This output argument is then
returned to the caller.

%\xs{1}

%\sy{1}	{close\_selected\_device}
\paragraph{close\_selected\_device : }

This function is called to close the connection to a device. First of all this
function calls the "dev\_free" routine of the device server library. It will 
then free the entry corresponding to that device in the "device\_table".

%\xs{1}

