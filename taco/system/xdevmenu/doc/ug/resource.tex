%****************************************************************************/
%
% File:		resource.tex
%
% Project:	xdevmenu application documentation
%
% Description:	"The resources file" section
%
% Author(s):	Faranguiss Poncet
%
% Original:	March 1991
%
%
% $Header: /home/jkrueger1/sources/taco/backup/taco/system/xdevmenu/doc/ug/resource.tex,v 1.1 2003-04-25 12:54:31 jkrueger1 Exp $
%
% Copyright (c) 1990 by European Synchrotron Radiation Facility, 
%                       Grenoble, France
%
%		       All Rights Reserved
%
%****************************************************************************/
The user is provided with an application default resources file named
"XDevmenu". However the user can modify any of the resources defined in that
file by redefining them in his own .Xdefaults file.

\sy{1}	{Default file location}

In most cases if the application is installed properly, this file
should be located on leo in the "/usr/lib/X11/app-defaults" directory.

If the file is not found there, check the environment variable "XAPPLRESDIR"
and the environment variable "LANG". If the variable "XAPPLRESDIR" is defined
correctly then make sure that the environment variable "LANG" is unset.

Note that a man page is available on leo, whisky and mars and can be helpful to set all the environment variables correctly.

\xs{1}

\sy{1}	{Customising the application}

The user can redefine any of the resources defined in the application default
file, but he should not try to modify the default resources file directly as
he has not the write access on that file. Instead, the user should define those
resources he wants to modify in his home login resources file, called usually
".Xdefaults".

\xs{1}

\sy{1}	{Some of the most important resources}

To know about all of the available resources, the user is invited to have a look
at the application default resources file. In this section we describe only
a subset of them.
\begin{itemize}
\item
{\bf xdevmenu*ch\_dev\_bullboard*ch\_dev\_loc\_text*value : }this resource
defines the device location. The default value is "/users/a/devices". It is
a very important resource because it allows the application to compute the real 
device name given its full pathname. For example, if the full pathname of
a device is "/users/a/devices/SY/V-PEN/S9-1" and if this resource value is
"/users/a/devices" then the real device name would be "SY/V-PEN/S9-1".
Notice that this resource defines the device location only at the start up of
the application, since the user can change it dynamically using the option
"modification".
\item
{\bf xdevmenu*ch\_dev\_bullboard*ch\_dev\_fil\_text*value : } this resource
defines a subdirectory for the device location. The default value for the 
device location being "/users/a/devices", the value given to the subdirectory
enables the application to compute the real device names according to a more
specific full pathname. For example if
xdevmenu*ch\_dev\_bullboard*ch\_dev\_loc\_text*value is "/users/a/devices" 
and xdevmenu*ch\_dev\_bullboard*ch\_dev\_fil\_text*value
 is "SY", only the device names beginning with "SY" will be computed. The default
value for the subdirectory filter is blank. The user can change dynamically
this resource using the option "modification".
\item 
{\bf xdevmenu*ANY\_DEVICE\_pixmap*labelPixmap : } this resource defines the
bitmap file name used in the icons representing all devices of type
"ANY\_DEVICE". This resource allows the user to define different icon images
depending on the type of the device being opened.
\end{itemize}

\xs{1}


