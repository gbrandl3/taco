%****************************************************************************/
%
% File:		uil_file.tex
%
% Project:	xdevmenu application documentation
%
% Description:	"The uil file (xdevmenu.uil)" section
%
% Author(s):	Faranguiss Poncet
%
% Original:	March 1991
%
%
% $Header: /home/jkrueger1/sources/taco/backup/taco/system/xdevmenu/doc/uil_file.tex,v 1.1 2003-04-25 12:54:30 jkrueger1 Exp $
%
% Copyright (c) 1990-2013 by European Synchrotron Radiation Facility, 
%                       Grenoble, France
%
%		       All Rights Reserved
%
%****************************************************************************/
The static part of the user interface is defined in this file. Therefore
we find in this file the description for~:
\begin{itemize}
\item
the main window\footnote[1]{OSF/MOTIF concept : see OSF/MOTIF programmer's
guide}, with the menu bar\footnotemark[1] and associated pulldown
menus\footnotemark[1],
\item
the file selection dialogue\footnotemark[1] used for "open device" function,
\item
the selection dialogue\footnotemark[1] used for the "execute command" function,
\item
the bulletinboard dialogue\footnotemark[1] used for the "option modification"
function,
\item
the row-column\footnotemark[1] widget used within the working area of the main
window, for the arrangement of the icons.
\item
the question dialogue window\footnotemark[1] used for confirming that a device 
has to be closed.
\item
the error dialogue window\footnotemark[1] used for the display of error messages
( ie  no selected device, device already opened .. ).
\end{itemize}

All these widgets are fetched at run time by the "xdevmenu.c" module.
The other parts of the interface~: the definition of the individual icons, and
the parameters prompting dialogue windows are created using Xm library without
using uil.

