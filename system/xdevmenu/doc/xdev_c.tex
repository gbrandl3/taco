%****************************************************************************/
%
% File:		xdev_c.tex
%
% Project:	xdevmenu application documentation
%
% Description:	"xdevmenu.c" section
%
% Author(s):	Faranguiss Poncet
%
% Original:	March 1991
%
%
% $Header: /home/jkrueger1/sources/taco/backup/taco/system/xdevmenu/doc/xdev_c.tex,v 1.1 2003-04-25 12:54:30 jkrueger1 Exp $
%
% Copyright (c) 1990-2014 by European Synchrotron Radiation Facility,
%                       Grenoble, France
%
%		       All Rights Reserved
%
%****************************************************************************/
In this module all the static part of the user interface is built (from the uil
file) and accessed.

Most of the callback procedures are defined in this file and the link is made
between the name of the callback in the uil file and the real name of the
function in the code.

This module contains the "main" function and in that way it calls Xtoolkit
to create the top most widget. 

There is not any specific data structure defined in this module.

%\sy{1}	{init\_application function}
\paragraph{init\_application function : }

This function is in charge of accessing the UID file (which is the result of
compilation of the UIL file) and to build the widgets using the Mrm library.

Before doing this task this function calls two other functions
"init\_device\_gui" and "init\_device\_interface" in order to initialize the
internal data structures of those modules.

%\xs{1}

%\sy{1}	{exec\_sel\_cmd function}
\paragraph{exec\_sel\_cmd function : }

This callback function is called when the user has selected a command from the
command list proposed for a device and asked for its execution.

This function tests if there is any input argument for the selected command.
If so a prompting dialogue window is displayed thanks to the "dev\_gui.c"
function "show\_cmd\_in\_param". 
If not the "exec\_sel\_cmd" function executes
the command thanks to the "dev\_interface" function "dev\_exec\_cmd". In both
cases the "exec\_sel\_cmd" function displays in the top part of the out and  
error window\footnote[1]{ OSF/MOTIF concept : see OSF/MOTIF programmer's guide } 
the output argument if any, else the date and the name of the executed 
command only. If the result of the command is an error received from the
device server, the latter is displayed in the bottom part of the same 
window.

%\xs{1}

