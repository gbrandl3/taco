%****************************************************************************/
%
% File:		appli_arch.tex
%
% Project:	xdevmenu application documentation
%
% Description:	"Application architecture" section
%
% Author(s):	Faranguiss Poncet
%
% Original:	March 1991
%
%
% $Header: /home/jkrueger1/sources/taco/backup/taco/system/xdevmenu/doc/appli_arch.tex,v 1.2 2003-12-17 14:24:25 jkrueger1 Exp $
%
% Copyright (c) 1990-2014 by European Synchrotron Radiation Facility,
%                       Grenoble, France
%
%		       All Rights Reserved
%
%****************************************************************************/
There are two kinds of architecture : external architecture and internal
architecture.

The external architecture shows the relationship of the application (as a whole)
with other external components (libraries, other processes).

The internal architecture shows the relationship of each of the internal
application modules with each other and also with the external world.

\subsubsection{External architecture}

This application will use the Motif libraries (Xm, Mrm) and Cern widgets
library (libwsc) for the user interface part and it will use the device server
libraries (libdsapi.a and libdbapi.a) for the operations concerning devices.

Figure 1 represents the external architecture:

\input arch_ext_pic

\subsubsection{Internal architecture}

Figure 2 represents the internal architecture :

\input arch_int_pic

The static description of the user interface is made in a UIL file. The C code
part of the application is composed of three modules :
\begin{itemize}
\item
{\bf xdevmenu.c : } the main module controlling the highest user interface part
(main menu-bar, dialog boxes). It calls the dev\_interface.c functions to
perform the device functions and the dev\_gui.c functions to perform pure user
interface input/output functions. It also calls the MOTIF libraries routines to
handle the user interface.
\item
{\bf dev\_interface.c : } device application interface module. This module
offers the routines to be called by the main module (and in some cases by the
dev\_gui module) in order to perform device operations.
This module in turn will call the device server functions to effectively 
perform these operations. This module represents the device
interface from the xdevmenu point of view.
\item
{\bf dev\_gui.c : } device graphical user interface module. This module provides
the means to display, to highlight a device as well as to remove it from the
screen. This module is in charge of graphical representation of the devices,
the commands and the command arguments.
It uses MOTIF libraries functions to provide this device graphical
representation.
\end{itemize}
